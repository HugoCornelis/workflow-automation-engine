\documentclass[a4paper,openbib,10pt]{article}
%\documentclass[a4paper,openbib,draft]{article}
%\usepackage[boxed]{algorithm}
\usepackage{afterpage}
%\usepackage{algorithmic}
\usepackage{alltt}
\usepackage{amsfonts}
\usepackage{amsmath}
\usepackage{amsthm}
\usepackage{boxedminipage}
\usepackage{calc}
\usepackage{color}

\usepackage{csquotes}
\renewcommand{\mkbegdispquote}[2]{\itshape}

\usepackage{endnotes}
\usepackage{enumerate}
%\usepackage{graphs}
\usepackage{graphicx}
\usepackage{indentfirst}
%\usepackage{ite}
\usepackage{makeidx}
%\usepackage{minitoc}
%\usepackage{multind}
%\usepackage{natbib}
%usepackage{wrapfig}

%\setlength{\textwidth}{15.7cm}
%\setlength{\textheight}{25cm}

%\setlength{\oddsidemargin}{0pt}

%\setlength{\topmargin}{0cm}
%\setlength{\headheight}{0cm}
%\setlength{\headsep}{0cm}
%\setlength{\topskip}{0cm}


%\setlength{\textwidth}{15.7cm}
%\setlength{\textheight}{24.2cm}

%\setlength{\oddsidemargin}{0pt}

%\setlength{\topmargin}{0cm}
%\setlength{\headheight}{1cm}
%\setlength{\headsep}{0.5cm}
%\setlength{\topskip}{0cm}

\setlength{\textwidth}{15.7cm}
\setlength{\textheight}{22cm}

\setlength{\oddsidemargin}{0pt}

\setlength{\topmargin}{0cm}
\setlength{\headheight}{1cm}
\setlength{\headsep}{0.5cm}
\setlength{\topskip}{0cm}

%\renewcommand {\baselinestretch} {1.5}

% \pagestyle{headings}

\newenvironment{treegraph}{\begin{framegraph}}{\end{framegraph}}
%\newenvironment{treegraph}{\begin{graph}}{\end{graph}}

\newenvironment{pagedtext}{\begin{minipage}}{\end{minipage}}
%\newenvironment{pagedtext}{\begin{boxedminipage}}{\end{boxedminipage}}

\newsavebox{\algobox}
\newlength{\algoenumpagewidth}

\newenvironment{algo-enumerate}[1][enumerate]{%
%  \sbox{\algobox}{#1}
  \setlength{\algoenumpagewidth}{\textwidth - 1.5cm}%
  \begin{pagedtext}[b]{\algoenumpagewidth}%
    \vspace{.5ex}%
%    \begin{\usebox{\algobox}}%
    \begin{enumerate}%
    }%
    {%
%    \end{\usebox{\algobox}}%
    \end{enumerate}%
    \vspace{.5ex}%
  \end{pagedtext}}

\newcommand{\PrincipalSerial}{\mathrm{ID}}
\newcommand{\PrincipalParent}{\mathrm{PA}}
\newcommand{\PrincipalDescendants}{\mathrm{DE}}
\newcommand{\PrincipalAncestors}{\mathrm{AN}}
\newcommand{\PrincipalSiblings}{\mathrm{SI}}
\newcommand{\PrincipalSiblingsLeft}{\mathrm{SI_L}}
\newcommand{\PrincipalSiblingsRight}{\mathrm{SI_R}}
\newcommand{\PrincipalChildren}{\mathrm{CH}}

\newcommand{\PrincipalSerialA}{\mathrm{ID}^{\mathcal{A}}}
%\newcommand{\PrincipalParentA}{\mathrm{PA}^{\mathcal{A}}}
\newcommand{\PrincipalDescendantsA}{\mathrm{DE}^{\mathcal{A}}}
%\newcommand{\PrincipalAncestorsA}{\mathrm{AN}^{\mathcal{A}}}
\newcommand{\PrincipalSiblingsA}{\mathrm{SI}^{\mathcal{A}}}
\newcommand{\PrincipalSiblingsLeftA}{\mathrm{SI^{\mathcal{A}}_L}}
\newcommand{\PrincipalSiblingsRightA}{\mathrm{SI^{\mathcal{A}}_R}}
%\newcommand{\PrincipalChildrenA}{\mathrm{CH}^{\mathcal{A}}}

\newcommand{\PrincipalSerialAi}{\mathrm{ID}^{\mathcal{A}_i}}
%\newcommand{\PrincipalParentAi}{\mathrm{PA}^{\mathcal{A}_i}}
\newcommand{\PrincipalDescendantsAi}{\mathrm{DE}^{\mathcal{A}_i}}
%\newcommand{\PrincipalAncestorsAi}{\mathrm{AN}^{\mathcal{A}_i}}
%\newcommand{\PrincipalSiblingsAi}{\mathrm{SI}^{\mathcal{A}_i}}
%\newcommand{\PrincipalSiblingsLeftAi}{\mathrm{SI_L}^{\mathcal{A}_i}}
%\newcommand{\PrincipalSiblingsRightAi}{\mathrm{SI_R}^{\mathcal{A}_i}}
%\newcommand{\PrincipalChildrenAi}{\mathrm{CH}^{\mathcal{A}_i}}

\newcommand{\PrincipalSerialB}{\mathrm{ID}^{\mathcal{B}}}
%\newcommand{\PrincipalParentB}{\mathrm{PA}^{\mathcal{B}}}
%\newcommand{\PrincipalDescendantsB}{\mathrm{DE}^{\mathcal{B}}}
%\newcommand{\PrincipalAncestorsB}{\mathrm{AN}^{\mathcal{B}}}
%\newcommand{\PrincipalSiblingsB}{\mathrm{SI}^{\mathcal{B}}}
%\newcommand{\PrincipalSiblingsLeftB}{\mathrm{SI^{\mathcal{B}}_L}}
%\newcommand{\PrincipalSiblingsRightB}{\mathrm{SI^{\mathcal{B}}_R}}
%\newcommand{\PrincipalChildrenB}{\mathrm{CH}^{\mathcal{B}}}

\newcommand{\Root}{\mathrm{Root}}

\newcommand{\Modelset}{%
  \begin{iteblock}(1ex,1.9ex)%
    \ITE(12.5 16.5 180 0.6) $\Delta$%
    \ITE(-5 0 0 0.6) $\Delta$%
  \end{iteblock}%
}


\newtheorem{definition}{Definition}[section]
\newtheorem{property}{Property}[section]
\newtheorem{algo}{Algorithm}[section]

\makeindex

\title{G-3 Workflow Automator
  \\
  {\tiny Document version: \pdfmdfivesum file {\jobname}} }

\author{Hugo Cornelis}
\begin{document}

\definecolor{white}{rgb}{1,1,1}
\definecolor{grey1}{rgb}{0.1,0.1,0.1}
\definecolor{grey2}{rgb}{0.2,0.2,0.2}
\definecolor{grey3}{rgb}{0.3,0.3,0.3}
\definecolor{grey4}{rgb}{0.4,0.4,0.4}
\definecolor{grey5}{rgb}{0.5,0.5,0.5}
\definecolor{grey6}{rgb}{0.6,0.6,0.6}
\definecolor{grey7}{rgb}{0.7,0.7,0.7}
\definecolor{grey8}{rgb}{0.8,0.8,0.8}
\definecolor{grey9}{rgb}{0.9,0.9,0.9}
\definecolor{black}{rgb}{0,0,0}
\definecolor{red}{rgb}{1,0,0}
\definecolor{red1}{rgb}{0.8,0,0}
\definecolor{red2}{rgb}{0.7,0,0}
\definecolor{red3}{rgb}{0.5,0,0}
\definecolor{green}{rgb}{0,1,0}
\definecolor{blue}{rgb}{0,0,1}

\maketitle
%\tableofcontents
%\newpage

\theoremstyle{plain}

\section{Document Purpose}

Provide an overview of the Genesis 3.0 project workflow automator, its
configuration and its use.


\section{Introduction}

Software development on Linux systems is often complemented with the
implementation and use of project-specific shell commands.  During
project life-time command use is fine-tuned with several options.
Several sophisticated commands are collected in project-specific shell
scripts.

% These
% scripts are developed and enhanced during the project life-time and after

% comprehensive the workflows from https://gitlab.com/essensium-mind/abb/workflow-scripts look at first sight.

% I can imagine it must really help in reproducing the same results 6
% months later without having to remember all the bells and whistles
% specific to each project!

These shell scripts often have the following problems:

\begin{itemize}
\item The scripts are often so sophisticated that they are difficult
  to enhance (missing modularity).
\item After initial implementation and use they are often forgotten.
  Using these scripts after a several months of project inactivity has
  the same complexity as implementing them.
\item The relationships between scripts of the same project is often
  unclear and undocumented.  Any hierarchy between the scripts is at
  most implicit.
\item The scripts are often specific to one user which hinders
  knowledge sharing of technical aspects of the project.
\item There is no single starting point for the use of the scripts
  supporting a common project.
\end{itemize}

\vspace{4mm}

The G-3 project workflow automator addresses these problems by:

\begin{itemize}
\item Separation between imperative style commands and declarative
  configuration.
\item Providing a hierarchy of scripts with common {\it workflow
    targets} and specific {\it commands} for operating on these
  targets.
\item Providing completion at the shell command line for {\it workflow
    targets}.
\item Providing completion at the shell command line for {\it workflow
    configuration}.
\item Color support for project specific keywords.
\item Generating overviews of command sequences, similar to UML role
  diagrammas.
\end{itemize}

\section{An Example}


\section{Installation}

The workflow automation engine of the Genesis 3.0 project (G-3) is
available from {\bf GitHub}:

\begin{verbatim}
git clone https://github.com/HugoCornelis/developer.git
\end{verbatim}

After cloning this repository it is installed with {\tt configure \&\&
  make \&\& sudo make install}

For expediency a symbolic link to the workflow engine in /usr/local/bin/workflow is created in the bin directory in the user’s home directory.
The automation scripts and their configuration are available from the git repository at:

\section{Starting a new Project}



%\newpage

%\listoffigures


%\listoftables

%\listofalgorithms

% \bibliographystyle{unsrt}

% \bibliography{./bibliography/5g-covid}

%\printindex

{\small Document version: \pdfmdfivesum file {\jobname}}

\end{document}

